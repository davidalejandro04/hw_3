\documentclass[letterpaper,12pt]{article}
\usepackage{tabularx} 
\usepackage{amsmath}  
\usepackage{graphicx} 
\usepackage[margin=1in,letterpaper]{geometry} % decreases margins


\begin{document}

\title{Resultados Tarea 3}
\author{David Alejandro Pabon Correa}
\date{\today}
\maketitle

\section{Ecuacion de onda en dos dimensiones}

A continuacion se presentan las graficas de $\Phi(t,x,y)$ para t=30 y t=60.
\begin{figure}[ht] 
	\includegraphics[width=\linewidth]{t30.png}

		 \caption{
        	        \label{fig:exp_plots}  
            	    $\Phi(30,x,y)$
        }
\end{figure}


\begin{figure}[ht] 

	\includegraphics[width=\linewidth]{t60.png}

		 \caption{
        	        \label{fig:exp_plots}  
            	     $\Phi(60,x,y)$.
        }
\end{figure}



\begin{figure}[ht] 

\section{Sistema Solar}
A continuacion se muestra una grafica de las orbitas de los planetas. 
	\includegraphics[width=\linewidth]{planetas.png}

		 \caption{
        	        \label{fig:exp_plots}  
            	    Orbitas de los planetas en el sistema solar.
        }
\end{figure}

\end{document}
